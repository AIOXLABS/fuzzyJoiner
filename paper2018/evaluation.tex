\section{Experiments}
\label{results}
In building the models, we employed early stopping using the usual metric of accuracy of the validation data, and we performed hyper-parameter tuning using grid search varying margins for our loss function from 1-20.  Accuracy was defined as the percentage of validation triples where positive distances from anchor were less than negative distances from anchor.  For all our runs, test accuracy as measured by this metric ranged in the 9-- range.  Table \ref{Evaluation} shows the results for the people and company datasets, run with a fixed $k$ of 20 neighbors.  Because we compared across different losses, the results are categorized by each loss function for each dataset we tested.  For all the results reported here, we ran multiple runs because of the stochastic nature of neural network models; the results here are representative of what we saw across runs. 

We report multiple metrics to measure the effectiveness of training, some of which are not standard because of the experimental setup, so we define them below:
\begin{itemize}
\item \textbf{Recall}.  Recall is measured by the percentage of positives in the nearest neighbor set of each anchor.
\item \textbf{Precision@1}.  Precision@1 is measured by the percentage of anchors with the very nearest neighbor being a positive.
\item \textbf{Precision}.  Precision is measured by the percentage of all positives for all anchors that were closer to the anchor than any other negative.
\end{itemize}
 
\begin{table}[ht]
\caption{Precision and Recall by loss functions}
\label{Evaluation}
\begin{center}
%{\scriptsize
\begin{tabular}{|l|l|r|r|r|}
\hline
Entities & Loss & Recall & \multicolumn{2}{|c|}{Precision} \\
 & & & @1 & All \\
\hline
\multirow{4}{*}{People} & Adapted & .81 & .81 & .63 \\
\cline{2-5}
& Triplet & .74 & .84 & .55 \\
\cline{2-5}
& Improved & .71 & .52 & .45 \\
\cline{2-5}
& Angular & .04 & .09 & .02 \\
\hline
\multirow{4}{*}{Corps} & Adapted & .72 & .73 & .64 \\
\cline{2-5}
& Triplet & .75 & .76 & .67 \\
\cline{2-5}
& Improved & .74 & .72 & .64 \\
\cline{2-5}
& Angular & .26 & .32 & .22 \\
\hline
\end{tabular} %}
\end{center}
\end{table}


\begin{table}[ht]
\caption{Distance estimates after training}
\label{Evaluation}
\begin{center}
%{\scriptsize 
\begin{tabular}{|l|l|r|r|r|r|}
\hline
Entities & Loss & \multicolumn{2}{|c|}{Pos} & \multicolumn{2}{|c|}{Neg} \\
& & Mean & Std & Mean & Std \\
\hline
\multirow{4}{*}{People} & Adapted & 1.59 & 4.04 & 2.44 & 2.45 \\
\cline{2-6}
& Triplet & .46 & .18 & .55 & .10 \\
\cline{2-6}
& Improved & .15 & .38 & .21 & .18 \\
\cline{2-6}
& Angular & .51 & .25 & .06 & .02 \\
\hline
\multirow{4}{*}{Corps} & Adapted & 2.38 & 3.77 & 3.23 & 1.71 \\
\cline{2-6}
& Triplet & .39 & .28 & .55 & .09 \\
\cline{2-6}
& Improved & .32 & .50 & .42 & .19 \\
\cline{2-6}
& Angular & .07 & .04 & .04 & .01 \\
\hline
\end{tabular}%}
\end{center}
\end{table}


\subsection{Performance for Joins}

 For fuzzy joins, we need both precision and recall to be high.  Without good recall, a join will potentially miss names that should be joined.  Without high precision, a join will mistakenly join many inappropriate names.  To quantify recall, we measure, for each loss function.  The numbers for our adapted loss function are 81\% for people and 72\% for companies, so a join could capture most similar names.

 Precision is a little trickier to define, but one way is to measure how many true matches we get in the neighborhood before seeing a single mistake, i.e. a match that should not be there.  That metric gives a picture of how many names would be correctly found, on average, by a join.  Our loss function gives 62\% by this metric for people and 64\% for companies, so about two thirds of recalled names would be found before finding a single error.  Since every name has at least one other name for the same entity, we also measure precision at 1, which is the probability that the very nearest neighbor is a true match.  For that, our loss yields 81\% for people and 73\% for companies.  Note that measuring precision for larger neighborhoods is tricky, since not every name has more than one name for the same entity.

\subsection{Learning Performance}

 We also assessed our learning mechanism more directly by examining how the nearest neighborhood changed from before to after training.  As can be seen from Table \ref{Evaluation}, recall is improved greatly, with the bigger change for people, in which case it improves from 3\% to 80\%.  For companies, it is from 17\% to 72\%.  Thus training is clearly effective in moving actual names for the same entity into the nearest neighbors.  For precision, the fraction of true matches found before the first error improves from 16\% to 73\% for people and 16\% to 64\% for companies.  Precision at one improves from 10\% to 81\% for people and from 26\% to 73\% for companies.  In both these cases, before training the results are basically noise even though embeddings are used for context, but training has made the results usable.

\subsection{Comparison with Semi-Hard Negatives}

 We hypothesize that training against hard negatives can potentially benefit on datasets with characteristics like those of our people dataset, when compared to training against semi-hard negatives.  Such datasets have large numbers of hard negatives, as suggested by Figure~\ref{people_characteristics}, which shows positive distances higher on average than negative ones.  Our strategy of focusing on the hardest negatives, i.e. those in the nearest neighborhood, ought to learn a nearest neighborhood with fewer negatives.  The common technique of using semi-hard negatives will focus more on negatives that are further than already-distant positives, and hence place less emphasis on the nearest neighborhood.

We compare our results to those from training againt semi-hard negatives.  To compute semi-hard negatives, we took, for each entity, all its positives, and found all negatives in the nearest neighborhood of that positive that are further from the entity than is the positive.  We made triplets for each such pair of positive and negative.  We thus chose the hardest semi-hard triplets: the negatives are as close to the positive as possible while still being further from the entity.  We ran experiments again for adapted loss on our people dataset, changing only the training used; these results are in Table~\ref{hard-semi-hard}.

\begin{table}[ht]
\caption{Precision and Recall for hard and semi-hard training}
\label{hard-semi-hard}
\begin{center}
%{\scriptsize
\begin{tabular}{|l|r|r|r|}
\hline
Training & Recall & Precision@1 & Precision \\
\hline
hard & .81 & .81 & .63 \\
\hline
semi-hard & .63 & .61 & .43 \\
\hline
\end{tabular}%}
\end{center}
\end{table}

 The results show training on hard negatives produces consistently better results, both for precision and recall.  Hard negative training resulted in recall of 81\% of posiives in the nearest set versus 63\% for semi-hard negatives.  Precision@1 is similar, with 81\% for hard negatives versus 61\% for semi-hard.  Overall precision, defined as the fraction of positives closer than any negative is 63\% for hard negatives versus 43\% for semi hard.

\subsection{Generalization Test on Faces}

 We have demonstrated that our strategy for joins could work well for textual names of people and companies, but the technique could potentially work for any kind of data for which a vector embedding can be made.  To test how well that works, we evaluated two existing models for face recognition that were trained with the same approaches defined in the \cite{DBLP:conf/cvpr/SchroffKP15} on two different datasets, VGGFace2 \cite{DBLP:conf/fgr/CaoSXPZ18}, and CasiaWebFace \cite{DBLP:conf/cvpr/SchroffKP15}.  We took the existing trained model at \url{https://github.com/davidsandberg/facenet}, and extracted the output embeddings for faces from the LFW test set \cite{Huang2012a} using these embeddings.  We put these output embeddings into an ANN structure, and computed our metrics on that.  Recall of the same face was remarkably similar to our results for names, getting 80\% on the VGGFace2 dataset, and 76\% for the Casia Web Face dataset.  However, we found precision at one was much lower on both datasets at 23\% for VGGFace2 and 21\% for Casia Web faces.

We hypothesize that one reason for lower precision at one is the nature of the training.  Training on pre-built LFW model focused on what they call semi-hard negatives, which are triplets for which the negative is further from training anchor than the corresponding positive.  Thus, negatives which are very close to an anchor may get less training to move them away from that anchor.  Our training focuses on nearby negatives, which allows them to be moved.  This is clearly a hypothesis that needs to be tested further.

\subsection{Discussion}

 Our results and our comparison with faces seems to show that training on hard negatives does not compromise recall overall and can boost precision for near neighbors.  Since joins will be most effective and efficient by focusing on a small, precise neighborhood, training on hard negatives may in fact be a useful addition.
